\documentclass[10pt]{article}
\usepackage{hyperref}
\usepackage{graphicx}
\usepackage[margin=2cm]{geometry}
\usepackage{amsmath}
\usepackage{booktabs}
\renewcommand{\Re}{\operatorname{Re}}
\renewcommand{\Im}{\operatorname{Im}}
\author{Shao-yu Tseng}
\date{\today}
\renewcommand{\thesubsection}{\arabic{subsection}}
\renewcommand{\thesection}{}
\title{Homework \#4}
\begin{document}

\maketitle

\setlength\parindent{0pt}
\section{Random Numbers}

\subsection{Uniform distribution histograms}
\includegraphics[width=0.8\textwidth]{rnumbers-part1.png}
\subsection{Gaussian distribution histograms}
This implements the Box-Muller transformation algorithm.
We can observe that the probability density function is close to the Gaussian pdf, especially with 1000000 samples.

\includegraphics[width=0.8\textwidth]{rnumbers-part2.png}

\section{2D Random Walk}
\subsection{}
Averaging across 10000 walks:

\includegraphics[width=0.8\textwidth]{rwalk-part1.png}
\subsection{}
We can observe from the second part of the plot that \(<x^2>\) grows linearly with respect to time.

\section{Diffusion equation}
\subsection{}

Using integral table for  \(\int^{\infty}_{-\infty} x e^{-a x^{2}} = \frac{1}{2}\sqrt{\frac{\pi}{a^{3}}}\)

\begin{align*}
\langle x(t)^2 \rangle &= \int_{-\infty}^{\infty} x^2 \rho(x, t) dx\\
&= \int_{-\infty}^{\infty} \frac{x^2}{\sqrt{2 \pi \sigma(t)^{2}}} \exp(-\frac{x^{2}}{2\sigma(t)^{2}})  dx\\
&= \frac{1}{\sqrt{2 \pi \sigma(t)^{2}}}  \int_{-\infty}^{\infty} x^{2} \exp(-\frac{x^{2}}{2\sigma(t)^{2}})  dx \\
&= \frac{1}{\sqrt{2 \pi \sigma(t)^{2}}} \frac{1}{2}\sqrt{\frac{\pi}{\frac{1}{2\sigma(t)^{2}}^{3}}} \\
&= \frac{1}{\sqrt{2 \pi \sigma(t)^{2}}} \frac{1}{2} \sqrt{\pi} \sqrt{2\sigma(t)^{6}} \\
&= \frac{1}{\sqrt{2 \pi}} \frac{1}{\sigma(t)} \sqrt{\pi} \sqrt{2} \sigma(t)^{3} = \sigma(t)^{2}
\end{align*}
\subsection{}
\includegraphics[width=0.8\textwidth]{diffusion-part2.png}

We can observe that the Gaussian almost perfectly overlaps the distribution (the blue curve is under the orange). This is because a spreading Gaussian is the solution to the diffusion equation given a point-like source.

\section{Mixing of two gases}
\subsection{}
Plotted together with part 2.
\subsection{}
\includegraphics[width=0.7\textwidth]{gases-part2.png}

\subsection{}
As we can observe in the plot below, the particles slowly move to reduce the concentration gradient.

\includegraphics[width=0.7\textwidth]{gases-part3.png}


\section{Contributions}
Expression for the finite difference method for solving the diffusion equation.

\url{https://sites.me.ucsb.edu/~moehlis/APC591/tutorials/tutorial5/node3.html}

Integral tables

\url{https://en.wikipedia.org/wiki/List_of_integrals_of_exponential_functions}
\end{document}
